\documentclass{article}
\usepackage{cctexexample}

\title{\LaTeX{} font guide}
\author{Carleton College \LaTeX{} workshop}
\date{}

\newcommand*{\code}[1]{\texttt{#1}}
\newcommand*{\filename}[1]{\texttt{#1}}
\newcommand*{\inst}[1]{\textbf{\boldmath{}#1}}

\usepackage{pdfpages}

\begin{document}
\maketitle

\LaTeX{} interacts with fonts differently than most software you are accustomed to.
It does not natively support using system fonts; you can't just choose Comic Sans from a drop-down menu and use it in your document.
(This is probably for the best.)

Nevertheless, \LaTeX{} does support a variety of fonts.
This document will showcase a few that are of high quality and have excellent math support.

\section*{Using a font in your document}
To use one of these fonts in a document using one of the Carleton templates, go to the preamble and look for these lines:
\begin{verbatim}
% The Latin Modern font is a modernized replacement for the classic
% Computer Modern. Feel free to replace this with a different font package.
\usepackage{lmodern}
\end{verbatim}
To switch to a different font package, just \emph{replace} \code{lmodern} with the name of that package.
(It is very important not to load multiple font packages; they can conflict and cause strange behavior.)

For example, to us the \code{kpfonts} package, you should replace the third line of the above with the following:
\begin{verbatim}
\usepackage{kpfonts}
\end{verbatim}

\includepdf{lmodern.pdf}
\includepdf{charter.pdf}
\includepdf{kpfonts.pdf}
\includepdf{fouriernc.pdf}
\includepdf{libertine.pdf}
\includepdf{utopia.pdf}
\includepdf{newtx.pdf}
\includepdf{newpx.pdf}
\end{document}