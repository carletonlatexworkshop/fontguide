\section*{Example text}

\begin{theorem*}[P\'{o}lya enumeration theorem, unweighted]
  Let $\Gamma$ be a group of permutations of a finite set $X$ of ``objects'' and let $Y$ be a finite set of ``colors''.
  Then the number of orbits under $\Gamma$ of $Y$-colorings of $X$ is given by
  \begin{equation*}
    \abs*{Y^{X} / \, \Gamma} = \frac{1}{\abs{\Gamma}} \sum_{\gamma \in \Gamma} t^{c \pbrac{\gamma}}
  \end{equation*}
  where $t = \abs{Y}$ and where $c \pbrac{\gamma}$ is the number of cycles of $\gamma$ as a permutation of $X$.
\end{theorem*}

\begin{theorem*}[Green's theorem, two dimensions]
  Let $C$ be a positively-oriented, piecewise-smooth, simple closed curve in the plane.
  Let $D$ be the region bounded by $C$.
  If $L$ and $M$ are functions defined on an open region containing $D$ and have continuous partial derivatives, then
  \begin{equation*}
    \oint_{C} \pbrac*{L \, d x + M \, d y} = \iint_{D} \pbrac*{\frac{\partial M}{\partial x} - \frac{\partial M}{\partial y}} \, dx \, d y.
  \end{equation*}
\end{theorem*}

$A B \Gamma \Delta E Z H \Theta I K \Lambda M N \Xi O \Pi P \Sigma T Y \Phi X \Psi \Omega$

$\alpha \beta \gamma \delta \epsilon \zeta \eta \theta \iota \kappa \lambda \mu \nu \xi o \pi \rho \sigma \tau \upsilon \phi \chi \psi \omega$

Math numerals:
$1 2 3 4 5 6 7 8 9 0$

Text numerals:
1 2 3 4 5 6 7 8 9 0

Set notation: $\cbrac*{x \in \Xi \middle\vert f(x) \subseteq F^{-1} (\zeta)}$